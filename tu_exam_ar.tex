\documentclass[addpoints,12pt]{exam}
\usepackage{graphicx}
\usepackage[no-math]{fontspec}
\usepackage{url,amsmath}
\usepackage[quiet]{polyglossia}
\usepackage{amsmath,amsthm,amsfonts}
\usepackage{setspace}
\usepackage{paralist}
\usepackage[geometry]{ifsym}
\usepackage{float}
\usepackage{fancybox}	
\usepackage{calc}
\usepackage{xcolor}
\usepackage{hyperref}
\usepackage{amssymb}
\usepackage{tikz}
\usepackage{fontspec}% هذه لنستطيع تغيير الخطوط
\usepackage{afterpage}

\setlength{\topmargin}{1.5cm}
\usepackage[lmargin=3.0cm, rmargin=3.0cm,tmargin=2.50cm,bmargin=2.50cm]{geometry}
\setdefaultlanguage{arabic} % Remove  [numerals=maghrib]
\setotherlanguage[variant=british]{english}
\setmainfont{AL-Mohanad}
\setromanfont{Junicode}
\defaultfontfeatures{Scale=MatchLowercase}
\setmonofont{AL-Mohanad}
\setsansfont{AL-Mohanad}
\newfontfamily\arabicfont[Script=Arabic,Scale=1]{Amiri}
\newfontfamily\arabicfonttt[Script=Arabic,Scale=1.3]{Times New Roman}
\newcommand{\margin}[1]{
\reversemarginpar
\marginpar{\fbox{#1}}}
\newcommand*\mycirc[1]{%
  \begin{tikzpicture}
    \node[draw,circle,inner sep=1pt] {#1};
  \end{tikzpicture}}

\begin{document}
\pagestyle{headandfoot}
\firstpageheadrule
\firstpageheader{Name:} { } {التحليل العددي}
\firstpagefooter{}{الإختبار النهائي, صفحة رقم \thepage\ من \numpages}
{}
\runningheader{Name:} { } {التحليل العددي}
\runningfooter{}{صفحة رقم \thepage\ من \numpages}
{}
\runningheadrule
\runningfootrule

%%%%% Cover page

\coverfirstpageheader{جامـــعة الطائــــــف \\ قسم الرياضيات و الإحصاء \\ \& ش}{\includegraphics[scale=0.5]{logo1.jpg}}{التحليل العددي \\ الإختبار النهائي \\ الزمن: 3 ساعات}
\runningheadrule
\runningfootrule
\onehalfspacing
\newcommand{\tf}[1][{}]{%
\fillin[#1][0.25in]%
}

\[ \]
\begin{center}
\shadowbox{\begin{minipage}{5in}
 الإسم:
 \dotfill\\
الرقم الجامعي:
\dotfill
\end{minipage}}
\end{center}
\begin{center}
\doublebox{\begin{minipage}{4.5in}
\begin{center}
\textbf{
تعليمات هامة:\\
}
\end{center}
1- أجب عن  الأسئلة في نفس الورقة.\\
2- تأكد من كتابة إسمك أعلى كل ورقة في المكان المخصص لذلك.\\
3- تأكد أن عدد الصفحات هو 9 صفحات. \\
4- تم وضع مساحة كافية للإجابة، إذا إحتجت إلى مساحة إضافية للحل فـأكتب خلف الورقة. مع الإشارة إلى رقم السؤال.
\end{minipage}}
\end{center}
\mycirc{ التسلسلي الرقم}
%\begin{table}[h!h]
%\centering
%\begin{tabular}{||c||p{4cm}||p{4cm}|}
%\hline
%\hline
%&\\
% الســـؤال &   الدرجــــة\\
%&\\
%\hline
%&\\
%الأول  &  \\
%&\\\hline& \\
%الثاني & \\
%&\\\hline& \\
%الثالث & \\
%&\\\hline& \\
%الرابــع & \\
%&\\\hline& \\
%الخامــس & \\
%&\\
%&\\\hline& \\
%السادس & \\
%&\\
%\hline
%\hline
%& \\
%المجموع الكلــي & \\
%\hline
%\hline
%\hline
%\end{tabular}
%\end{table}
%\clearpage

\begin{questions}
\question[5]
\begin{enumerate}
\item[أ - ] ضع علامة ( $\surd$) أمام العبارة الصحيحة  وعلامة ( X ) أمام الخاطئة مع تصحيح الخطأ:
%\margin{10 درجات}
\begin{enumerate}
\item[1 - ]  لتكوين كثيرة حدود تايلور من الدرجة $n$ نحتاج إلى عدد $n$ من النقاط. \hspace{\stretch{0.1}} $\bigl( \:\:\:\: \bigr)$
\item[]\dotfill
\item[2 - ] في طريقة التنصيف يمثل المقدار $\displaystyle (\frac{b-a}{2})^n$ الحد الأعلى للخطأ المطلق. \hspace{\stretch{0.1}} $\bigl( \:\:\:\: \bigr)$
\item[]\dotfill
\item[3 - ] كلما زادت درجة كثيرة الحدود الإستكمالية كلما حصلنا على حل أدق. \hspace{\stretch{0.1}} $\bigl( \:\:\:\: \bigr)$
\item[]\dotfill
\item[4 - ] إذا كانت  $f(x)$ دالة متصلة على الفترة $[x_0, x_0+h]$، فإن الخطأ المرتكب عند إستخدام صيغ الفروق الأمامية لحساب قيمة $f'(x_0)$ يتناسب مع $h^2$. \hspace{\stretch{0.1}} $\bigl( \:\:\:\:
\bigr)$
\item[]\dotfill
%\item[5 - ]  عند حل معادلة تفاضلية عدديا، فإن الخطأ المحلي لايكون بالضرورة متساويا عند كل نقاط التقسيم للمتغير المستقل. \hspace{\stretch{0.1}} $\bigl( \:\:\:\: \bigr)$
\item[5 - ] قيمة الخطأ المطلق تتناسب عكسيا مع الخطأ النسبي. \hspace{\stretch{0.1}} $\bigl( \:\:\:\: \bigr)$
\item[]\dotfill
\end{enumerate}
\hrulefill
\question
إعتبر الدالة $ f(x) =  e^{\sin(x)}$.  فإن كثيرة حدود تايلور من الدرجة الثالثة $p(x)$ حول $x_0 = 0$ تأخذ الشكل التالي
%%%%%%%%%%%%%%%%%%%%%%%%%%%%
\begin{center}
$\blacklozenge$$\blacklozenge$$\blacklozenge$\textbf{\textbf{  مع تمنياتي للجميع بالتوفيق والنجاح}} $\blacklozenge$$\blacklozenge$$\blacklozenge$
\end{center}

\end{document} 